\couplet{
	L’heure était venue,\\*
	où l’airain sacré,\\*
	de sa voix connue,\\*
	annonçait l’Ave.
}

\refrain{
	Ave, ave, ave Maria ! \vbis
}

\couplet{
	D’une main discrète,\\*
	l’ange, la prenant,\\*
	conduit Bernadette\\*
	au bord du torrent.
}

\couplet{
	Un souffle qui passe\\*
	avertit l’enfant\\*
	qu’une heure de grâce\\*
	sonne en ce moment.
}

\couplet{
	Sur Massabielle,\\*
	son œil voit soudain\\*
	l’éclat qui révèle\\*
	l’Astre du matin.
}

\couplet{
	C’est un doux visage,\\*
	rayonnant d’amour,\\*
	qu’entoure un nuage\\*
	plus beau que le jour.
}

\couplet{
	Son regard s’inspire\\*
	d’un reflet divin,\\*
	mais un doux sourire\\*
	dit : « Ne craignez rien. »
}

\couplet{
	Elle a la parure \\*
	d’un lys immortel ;\\*
	Elle a pour ceinture\\*
	un ruban du ciel.
}

\couplet{
	On voit une rose\\*
	sur ses pieds bénis,\\*
	fraîchement éclose\\*
	dans le Paradis.
}

\couplet{
	On voit un rosaire\\*
	glisser dans sa main,\\*
	et de la prière\\*
	tracer le chemin.
}

\couplet{
	L’âme palpitante,\\*
	le cœur enivré,\\*
	l’heureuse Voyante\\*
	répétait : Ave !
}

\couplet{
	L’extase s’achève,\\*
	le monde revient ;\\*
	l’enfant se relève disant :\\*
	« A demain ! »
}

\couplet{
	Avant chaque aurore,\\*
	son cœur en éveil\\*
	par soupirs implore\\*
	l’heure du réveil.
}

\couplet{
	« Mère de la terre,\\*
	ne défendez pas\\*
	d’aller voir la Mère\\*
	qui paraît là-bas ! »
}

\couplet{
	« Elle était si belle !\\*
	je veux la revoir...\\*
	Que désire-t-elle ?\\*
	Je veux le savoir. »
}

\couplet{
	Colombe fidèle\\*
	elle prend l’essor,\\*
	vole à tire d’aile\\*
	au nouveau Thabor.
}

\couplet{
	« Ô Dame chérie,\\*
	que demandez-vous ?\\*
	Parlez, je vous prie,\\*
	et dites-le nous ! »
}

\couplet{
	« Avec vos compagnes,\\*
	venez quinze fois\\*
	Près de ces montagnes\\*
	écouter ma voix. »
}

\couplet{
	« Enfant généreuse,\\*
	je vous le promets,\\*
	vous serez heureuse\\*
	au Ciel pour jamais. »
}

\couplet{
	« Si vous êtes bonne,\\*
	le monde est méchant ;\\*
	il ne me pardonne\\*
	de vous voir souvent. »
}

\couplet{
	« Le savant s’offense\\*
	de votre bonté ;\\*
	je n’ai pour défense\\*
	que la vérité. »
}

\couplet{
	Près de la Voyante,\\*
	au lever du jour,\\*
	la foule croyante\\*
	se rend tour à tour.
}

\couplet{
	La pauvre bergère,\\*
	comme un séraphin,\\*
	du ciel à la terre\\*
	franchit le chemin.
}

\couplet{
	La voilà ravie\\*
	dans cette Beauté\\*
	que le temps envie\\*
	à l’Éternité !
}

\couplet{
	De son blanc visage\\*
	les traits allongés\\*
	vers la sainte image\\*
	semblent emportés.
}

\couplet{
	Pendant sa prière\\*
	brille sur son front\\*
	la pure lumière\\*
	de la Vision.
}

\couplet{
	Le peuple fidèle\\*
	admire à genoux\\*
	de l’aube éternelle\\*
	le reflet si doux.
}

\couplet{
	« Qu’avez-vous, Madame ?\\*
	murmura l’enfant.\\*
	D’où vient que votre âme\\*
	est triste à présent ? »
}

\couplet{
	« Que faudra-t-il faire\\*
	pour tarir vos pleurs ? »\\*
	« Prier, dit la Mère,\\*
	pour tous les pécheurs. »
}

\couplet{
	« Je veux qu’ici même,\\*
	en procession,\\*
	le peuple qui m’aime\\*
	invoque mon nom. »
}

\couplet{
	« Que d’une chapelle,\\*
	le marbre béni\\*
	aux âges rappelle\\*
	mon séjour ici. »
}

\couplet{
	Ô profond mystère\\*
	d’un profond amour !\\*
	Faut-il qu’une mère\\*
	trahisse à son tour ?
}

\couplet{
	Deux fois Bernadette\\*
	vient aux lieux aimés ;\\*
	deux fois sur sa tête\\*
	les cieux sont fermés.
}

\couplet{
	« Ô Dame clémente !\\*
	ne savez-vous pas \\*
	qu’à votre Voyante, \\*
	on livre combats ? »
}

\couplet{
	« Enfant, prends courage\\*
	et bannis l’effroi ;\\*
	il faut que l’orage\\*
	éprouve la foi. »
}

\couplet{
	« Elle m’est rendue,\\*
	Elle reparaît ; \\*
	je goûte à sa vue\\*
	un nouvel attrait ! »
}

\couplet{
	« Vision chérie,\\*
	source de douceurs,\\*
	mettez, je vous prie,\\*
	comble à vos faveurs. »
}

\couplet{
	« On demande un gage\\*
	à votre bonté :\\*
	rendez témoignage\\*
	à la vérité. »
}

\couplet{
	« Que sur cette épine\\*
	et sous vos pieds\\*
	une fleur divine\\*
	pousse à l’églantier ! »
}

\couplet{
	Par un doux sourire\\*
	accueillant ces vœux,\\*
	Elle semble dire\\*
	« Je donnerai mieux.»
}

\couplet{
	La fleur éphémère\\*
	se dessèche et meurt ;\\*
	le cœur d’une mère\\*
	n’est point cette fleur.
}

\couplet{
	« À cette fontaine,\\*
	allez maintenant ;\\*
	l’eau dont elle est pleine\\*
	voilà mon présent. »
}

\couplet{
	L’enfant prend sa course\\*
	vers l’eau du torrent ;\\*
	un signe, à la source,\\*
	ramène l’enfant.
}

\couplet{
	Ses doigts, de la terre\\*
	déchirent le sein ;\\*
	d’humide poussière\\*
	elle emplit sa main.
}

\couplet{
	Fontaine de vie,\\*
	qui peut désormais,\\*
	de ton eau bénie,\\*
	compter les bienfaits ?
}

\couplet{
	« Et Vous dont la terre\\*
	admire le don,\\*
	céleste Étrangère\\*
	quel est votre nom ? »
}

\couplet{
	« À votre servante\\*
	qui prie à genoux,\\*
	à votre Voyante,\\*
	le cacherez-vous ? »
}

\couplet{
	Au cœur de sa Mère,\\*
	quatre fois l’enfant\\*
	d’une humble prière,\\*
	fait monter l’accent.
}

\couplet{
	Paraît cette fête,\\*
	où de Gabriel\\*
	l’Église répète\\*
	l’Ave solennel.
}

\couplet{
	La Beauté rayonne\\*
	d’un nouveau reflet\\*
	la Vierge abandonne\\*
	son dernier secret.
}

\couplet{
	À sa bien-aimée,\\*
	l’Apparition\\*
	de l’Immaculée\\*
	prononce le nom.
}

\couplet{
	Sainte Messagère,\\*
	remontez aux cieux ;\\*
	et de notre terre\\*
	portez-y les vœux !
}

\couplet{
	Vous vouliez du monde,\\*
	et de tous côtés,\\*
	il vient, il abonde,\\*
	il est à vos pieds.
}

\couplet{
	Salut, ô Vallée,\\*
	ô Trône d’amour,\\*
	où l’Immaculée\\*
	a pris son séjour !
}

\couplet{
	Avec son image,\\*
	avec ses bienfaits,\\*
	ta Grotte sauvage\\*
	n’est plus sans attraits.
}

\couplet{
	La fontaine y coule \\*
	sans jamais tarir\\*
	ainsi vient la foule\\*
	sans jamais finir.
}

\couplet{
	Pieux sanctuaire,\\*
	tu les vis présents,\\*
	de la France entière\\*
	les nobles enfants !
}

\couplet{
	Ta voûte sacrée,\\*
	depuis ce grand jour,\\*
	de chaque contrée\\*
	a vu le retour.
}

\couplet{
	Du trône de grâce\\*
	on sait le chemin :\\*
	le pèlerin passe\\*
	et passe sans fin.
}

\couplet{
	Heureux qui voyage\\*
	en ces lieux bénis !\\*
	On y prend passage\\*
	pour le Paradis.
}

\couplet{
	Astre salutaire,\\*
	que votre rayon\\*
	nous mène à la terre\\*
	de la Vision !
}
