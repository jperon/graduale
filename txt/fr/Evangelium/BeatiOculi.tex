\x~Lecture du Saint Évangile selon saint Luc.

En ce temps-là, Jésus dit à ses disciples : Heureux les yeux qui voient ce que vous voyez ! Car je vous déclare que beaucoup de prophètes et de rois ont voulu voir ce que vous voyez et ne l’ont pas vu, entendre ce que vous entendez et ne l’ont pas entendu. Et voilà qu’un docteur de la loi se leva pour le tenter, disant : Maître, que me faut-il faire pour posséder la vie éternelle ? Jésus lui dit : Qu’y a-t-il d’écrit dans la loi ? Comment lisez-vous ? Il répondit : Vous aimerez le Seigneur votre Dieu de tout votre cœur, de toute votre âme, de toutes vos forces et de tout votre esprit ; et votre prochain comme vous-même. Jésus lui dit : Vous avez bien répondu ; faites cela, et vous vivrez. Mais lui, voulant faire paraître qu’il était juste, dit à Jésus : Et qui est mon prochain ? Or Jésus, prenant la parole, dit : Un homme descendait de Jérusalem à Jéricho, et il tomba entre les mains des voleurs qui le dépouillèrent, et s’en allèrent après l’avoir couvert de coups, le laissant à demi mort. Or il arriva qu’un prêtre descendait par le même chemin, et l’ayant vu il passa outre. De même un lévite étant venu près du lieu, et le voyant, passa outre. Mais un Samaritain qui voyageait arriva près de lui, et, le voyant, fut ému de compassion. S’approchant donc, il banda ses blessures, versant dessus de l’huile et du vin ; et l’ayant mis sur son cheval, il le conduisit dans une hôtellerie où il prit soin de lui. Le lendemain il tira deux deniers qu’il donna à l’hôtelier en disant : Ayez soin de lui, et tout ce que vous dépenserez de plus, je vous le rendrai à mon retour. Lequel de ces trois vous parait avoir été le prochain de celui qui est tombé entre les mains des voleurs ? Le docteur répondit : Celui qui a exercé la miséricorde envers lui. Allez donc, lui dit Jésus, et faites de même.
